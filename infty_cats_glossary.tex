\documentclass{article}

% Language setting
% Replace `english' with e.g. `spanish' to change the document language
\usepackage[english]{babel}

% Set page size and margins
% Replace `letterpaper' with `a4paper' for UK/EU standard size
\usepackage[a4paper,top=2cm,bottom=2cm,left=0.4cm,right=0.4cm,marginparwidth=1.75cm]{geometry}
\usepackage{multirow}

% Useful packages
\usepackage{graphicx}
\graphicspath{ {./images/} }

%inserting a figure:

%\begin{figure}[h]
%\centering
%\includegraphics[width=0.5\textwidth]{pfigure_1}
%\caption{The compound pendulum}
%\end{figure}

\usepackage{amsmath}
\usepackage{enumitem}
\usepackage{amssymb}
\usepackage[colorlinks=true, allcolors=blue]{hyperref}
\usepackage{mathtools}
\usepackage{parskip}
\usepackage{mathrsfs}
\usepackage{tikz-cd}
\usepackage{stackengine}

\usepackage{longtable}
\usepackage{xcolor}

\newcommand{\surj}{\rightarrow\mathrel{\mkern-14mu}\rightarrow}

\newcommand{\xsurj}[2][]{%
  \xrightarrow[#1]{#2}\mathrel{\mkern-14mu}\rightarrow
}

\newcommand\xmono[2][]{\ensurestackMath{\mathrel{%
  \stackengine{1pt}{%
    \stackengine{0pt}{\rightarrowtail}{\scriptstyle#2}{O}{c}{F}{F}{S}%
  }{\scriptstyle#1}{U}{c}{F}{F}{S}%
}}}

\newcommand{\notimplies}{%
  \mathrel{{\ooalign{\hidewidth$\not\phantom{=}$\hidewidth\cr$\implies$}}}}

\newcommand{\centre}{\center}

\DeclareMathOperator{\Arg}{Arg}
\DeclareMathOperator{\can}{can}
\DeclareMathOperator{\charr}{char}
\DeclareMathOperator{\cod}{cod}
\DeclareMathOperator{\coev}{coev}
\DeclareMathOperator{\coker}{coker}
\DeclareMathOperator{\Div}{Div}
\let\div\relax
\DeclareMathOperator{\div}{div}
\DeclareMathOperator{\dom}{dom}
\DeclareMathOperator{\End}{End}
\DeclareMathOperator{\ev}{ev}
\DeclareMathOperator{\Ext}{Ext}
\DeclareMathOperator{\Fix}{Fix}
\DeclareMathOperator{\Frac}{Frac}
\DeclareMathOperator{\Gr}{Gr}
\DeclareMathOperator{\Hom}{Hom}
\DeclareMathOperator{\im}{im}
\DeclareMathOperator{\lk}{lk}
\DeclareMathOperator{\mor}{mor}
\DeclareMathOperator{\ob}{ob}
\DeclareMathOperator{\Open}{Open}
\DeclareMathOperator{\ord}{ord}
\DeclareMathOperator{\Pic}{Pic}
\DeclareMathOperator{\Rep}{Rep}
\DeclareMathOperator{\Sing}{Sing}
\DeclareMathOperator{\str}{str}
\DeclareMathOperator{\Sk}{Sk}
\DeclareMathOperator{\Spec}{Spec}
\DeclareMathOperator{\SPec}{Spec}
\DeclareMathOperator{\tr}{tr}
\DeclareMathOperator{\cvec}{Vec}

\def\done{\begin{flushright}\vspace{-4.35ex}\(\qed\)\end{flushright}}

\def\mlip{\ensuremath\abs f_\text{Lip}}
\def\summing{\ensuremath\sum^{\infty}_{n=1}}

\def\bb{\ensuremath\mathbb}
\def\la{\ensuremath\langle}
\def\ra{\ensuremath\rangle}
\def\notto{\ensuremath\nrightarrow}
\def\subq{\ensuremath\subseteq}
\def\id{\ensuremath\text{id}}
\def\inj{\ensuremath\hookrightarrow}
\def\xinj{\ensuremath\xhookrightarrow}
\def\comp{\ensuremath\mathbb{C}}
\def\real{\ensuremath\mathbb{R}}
\def\rat{\ensuremath\mathbb{Q}}
\def\inte{\ensuremath\mathbb{Z}}
\def\nat{\ensuremath\mathbb{N}}
\def\topo{\ensuremath\mathcal{T}}
\def\del{\ensuremath\partial}

\def\vv{\ensuremath\vec{v}}
\def\ee{\ensuremath\vec{e}}

\def\met{\ensuremath(X, d)}
\def\metx{\ensuremath(X, d_X)}
\def\mety{\ensuremath(Y, d_Y)}
\def\topox{\ensuremath(X, \mathcal{T}_X)}
\def\topoy{\ensuremath(Y, \mathcal{T}_Y)}

\def\nvert{\ensuremath\hspace{-0.5ex}\not\vert\hspace{0.5ex}} 

% correct spelling
\def\colour{\color}
\def\textcolour{\textcolor}

% \limits\sum or \sum\limits (one of them) will put the text under the sum
% [upquote=true] on \begin{lstlisting} will stop backticks being interpreted as quotes

\usepackage[backend=bibtex, style=alphabetic]{biblatex}
\addbibresource{References.bib}

\begin{document}
\DeclarePairedDelimiter{\norm}{\lVert}{\rVert} 
\DeclarePairedDelimiter{\abs}{\lvert}{\rvert} 
\DeclarePairedDelimiter{\ang}{\langle}{\rangle} 

\begin{centre}
\begin{longtable}{ |p{3.2cm}||p{5cm}|p{5.2cm}|p{5cm}|  }
 \hline
 Concept& 1-Categorical construction & \(\infty\)-Categorical construction &Intuition\\
 \hline\hline
 Accessible category & \(\mathcal{C}\) is locally small, admits \(\kappa\)-filtered colimits, and there is a set of \(\kappa\)-compact objects that generate the category under \(\kappa\)-filtered colimits. (\autocite{accessible}, Def 2.1) & \textcolour{red}{\(\mathcal{C}\) is locally small, admits \(\kappa\)-filtered colimits, the full subcategory \(\mathcal{C}^\kappa\subq \mathcal{C}\) of \(\kappa\)-compact objects is essentially small, and \(\mathcal{C}^\kappa\) generates \(\mathcal{C}\) under small, \(\kappa\)-filtered colimits. (\autocite{htt}, Prop 5.4.2.2)} & \textcolour{red}{[todo]}\\
 \hline
  \(F\)-Cartesian edge & \(F : X \to S \) a functor, \(f : x\to y \) a morphism in \(X\) is \(F\)-cartesian if the induced map \[X_{/f}\to X_{/y}\times_{S_{/F(y)}}S_{/F(f)}\] is an isomorphism of categories. (\autocite{cartesian}, Prop 2.4) & \(F : X \to S \) an inner fibration of simplicial sets, \(f : x\to y \) an edge in \(X\) is \(F\)-cartesian if the induced map \[X_{/f}\to X_{/y}\times_{S_{/F(y)}}S_{/F(f)}\] is a trivial Kan fibration. (\autocite{htt}, Def 2.4.1.1)& In the model structure on \textbf{sSet}, the fibrations are Kan fibrations and the weak equivalences are weak homotopy equivalences (\autocite{htt}, A.2.7). A trivial fibration in a model category is a map which is both a fibration and a weak equivalence, which in \textbf{sSet} is equivalent to the definition given in this table. Thus, being related by a Kan fibration is a higher categorical notion of `sameness'. \textcolour{red}{Why not a categorical equivalence? \autocite{htt} Rem 1.2.5.5 implies this is stronger, which would match more with the fact that the 1-categorical version is defined in terms of an isomorphism (not equivalence) of categories.} \\
\hline
 Category & Collection of objects \(\mathcal{C}\), set \(\Hom(X, Y)\) for every \(X, Y\in \mathcal{C}\), associative composition and identity morphisms & Simplicial set \(\mathcal{C} : \Delta^\text{op} \to \textbf{Set}\) satisfying the inner horn filling condition. (\autocite{htt}, Def 1.1.2.4)& Category with objects \(\mathcal{C}_0\), morphisms \(\mathcal{C}_1\), morphisms between morphisms \(\mathcal{C}_2\), etc. Inner horn filling defines a non-unique composition. \\
 \hline
 \(F\)-Cocartesian edge & \(F : X \to S \) a functor, \(f : x\to y \) a morphism in \(X\) is \(F\)-cocartesian if the induced map \[X_{f/}\to X_{x/}\times_{S_{F(y)/}}S_{F(f)/}\] is an isomorphism of categories. (\autocite{cartesian}, Prop 2.4) & \(F : X \to S \) an inner fibration of simplicial sets, \(f : x\to y \) an edge in \(X\) is \(F\)-cartesian if the induced map \[X_{f/}\to X_{x/}\times_{S_{F(y)/}}S_{F(f)/}\] is a trivial Kan fibration. (\autocite{htt}, Def 2.4.1.1 / Prop 2.4.1.8)& Note that the definitions of an inner fibration and a Kan fibration are invariant under taking opposites. For other intuition, see: \(F\)-cartesian edge. \\
\hline
 Colimit & A colimit for \(F : J \to \mathcal{C}\) is an initial cocone on \(F\). & A colimit for \(F : X\to \mathcal{C} \) (\(X\) a simplicial set, \(\mathcal{C}\) an \(\infty\)-category) is an initial object of \(\mathcal{C}_{F/}\). (\autocite{htt}, Def 1.2.13.4) & The obvious extension of the definition of the undercategory \(\mathcal{C}_{C/}\) for \(C : \{*\} \to \mathcal{C}\) to \(\mathcal{C}_{/F}\)  for an arbitrary functor \(F : J \to \mathcal{C}\) ends up being exactly \(\textbf{Cocone}(F)\). \\
\hline
\(\kappa\)-Compact object & Let \(C \in \mathcal{C}\), and let \(j_C : \mathcal{C} \to \textbf{Set}\) denote the functor represented by \(C\). If \(\mathcal{C}\) admits \(\kappa\)-filtered colimits, then \(C\) is \(\kappa\)-compact if \(j_C\) commutes with filtered colimits. (\autocite{htt}, 5.3.4) & \textcolour{red}{Let \(C \in \mathcal{C}\), and let \(j_C : \mathcal{C} \to \hat{\mathcal{S}}\) denote the functor represented by \(C\). If \(\mathcal{C}\) admits \(\kappa\)-filtered colimits, then \(C\) is \(\kappa\)-compact if \(j_C\) preserves \(\kappa\)-filtered colimits.\footnote{Lurie introduces the term \textit{\(\kappa\)-continuous} for such functors, but in ordinary category theory this generally means a functor which preserves \(\kappa\)-small limits; a functor preserving \(\kappa\)-filtered colimits is called \textit{\(\kappa\)-finitary}. I have thus steered clear of this term.} (\autocite{htt}, Def 5.3.4.5)} & \textcolour{red}{[todo]}\\
\hline
Dual object & \textcolour{red}{[todo]} & \textcolour{red}{[todo]} & \textcolour{red}{[todo]}\\
\hline
Essentially small category & \textcolour{red}{[todo]} & \textcolour{red}{[todo]} & \textcolour{red}{[todo]}\\
\hline 
Essentially surjective functor & \(F : \mathcal{C} \to \mathcal{D}\) is essentially surjective if for every \(D \in \mathcal{D}\), there exists some \(C \in \mathcal{C}\) with \(FC \cong D\). & \(F : \mathcal{C} \to \mathcal{D}\) is essentially surjective if h\(F :\) h\(C \to\) h\(\mathcal{D}\) is essentially surjective. (\autocite{htt}, Def 1.2.10.1) & Essentially surjective up to homotopy.\\
\hline
Faithful functor & \(F : \mathcal{C} \to \mathcal{D}\) is faithful if \(\Hom(X,Y)\to \Hom(FX, FY)\) is injective for all \(X,Y \in \mathcal{C}\). & \(F : \mathcal{C} \to \mathcal{D}\) is faithful if h\(F :\) h\(\mathcal{C} \to\) h\(\mathcal{D}\) is faithful. (\autocite{htt}, Def 1.2.10.1) & Faithful up to homotopy.\\
\hline
\(\kappa\)-Filtered category & For a regular cardinal \(\kappa\), \(\mathcal{C}\) is \(\kappa\)-filtered if, for every \(\kappa\)-small category \(J\) and every functor \(F : J \to \mathcal{C}\), there exists a cocone on \(F\). & For a regular cardinal \(\kappa\), \(\mathcal{C}\) is \(\kappa\)-filtered if, for every \(\kappa\)-small simplicial set \(X\) and every map \(f : X \to \mathcal{C}\), there exists a map \(\overline f : K^\rhd \to \mathcal{C}\) extending \(f\). (\autocite{htt}, Def 5.3.1.7) & A cocone on \(F\) is a collection of compatible maps \((\lambda_j : F(j) \to C)\). Define \(\overline F : J \star [0] \to \mathcal{C}\) to be \(F\) on \(J\), send the cone point to \(C\), and send the unique morphisms \(*_j\) from \(j \in J\) to the cone point to the \(\lambda_j\). Conversely, if you have some \(\overline F\) extending \(F\), define \(\lambda_j := F(*_j)\).\\
\hline
 Final object & Object \(C\in \mathcal{C} \) such that for any other object \(C' \in \mathcal{C}\), there exists a unique morphism \(C' \to C\). & Object \(C \in \mathcal{C}\) such that \(C\) is final in h\(\mathcal{C}\), regarded as an enriched category over \(\mathcal{H}\).  (\autocite{htt}, Def 1.2.12.1) & Object \(C\in \mathcal{C}\) such that for any other object \(C' \in \mathcal{C}\), there exists a unique (up to homotopy) morphism \(C' \to C\).\\
\hline
Full functor & \(F : \mathcal{C} \to \mathcal{D}\) is full if \(\Hom(X,Y)\to \Hom(FX, FY)\) is surjective for all \(X,Y \in \mathcal{C}\). & \(F : \mathcal{C} \to \mathcal{D}\) is full if h\(F :\) h\(\mathcal{C} \to\) h\(\mathcal{D}\) is full. (\autocite{htt}, Def 1.2.10.1) & Full up to homotopy.\\
\hline
Functor & Functor. & Natural transformation of simplicial sets. (\autocite{htt}, 1.2.7)&-\\
\hline
Groupoid & Category whose morphisms are all invertible. & Kan complex. & Not only can you find (non-unique) `composites', but you can also fill in diagrams like \(\begin{tikzcd}
C \arrow[r, "\id"] \arrow[d, swap, "f"]  & C  \\
D \arrow[ur, swap, dashrightarrow, ""]  &
\end{tikzcd}\)\;\; \(\begin{tikzcd}
C \arrow[r, "\id"] \arrow[d, swap, dashrightarrow]  & D  \\
C \arrow[ur, swap, "f"]  &
\end{tikzcd}\)
 \\
\hline
Initial object & Object \(C\in \mathcal{C} \) such that for any other object \(C' \in \mathcal{C}\), there exists a unique morphism \(C \to C'\). & Object \(C \in \mathcal{C}\) such that \(C\) is initial in h\(\mathcal{C}\), regarded as an enriched category over \(\mathcal{H}\).  (\autocite{htt}, Def 1.2.12.1) & Object \(C\in \mathcal{C}\) such that for any other object \(C' \in \mathcal{C}\), there exists a unique (up to homotopy) morphism \(C \to C'\).\\
\hline
 Join & \(\mathcal{C}\star \mathcal{D}\) has objects \(\ob \mathcal{C} \sqcup \ob \mathcal{D}\), and \(\Hom_{\mathcal{C}\star \mathcal{D}}(X,Y)\) is given by: \(\begin{cases}
\Hom_\mathcal{C}(X,Y) & X, Y \in \mathcal{C},\\
\Hom_\mathcal{D}(X,Y) & X,Y \in \mathcal{D},\\
\emptyset & X \in \mathcal{D}, Y \in \mathcal{C},\\
* & X \in \mathcal{C}, Y\in \mathcal{D}.
\end{cases}\) (\autocite{htt}, 1.2.8) & \(\mathcal{C}\star \mathcal{D} \) has \(n\)-simplicies \((\mathcal{C} \star \mathcal{D})=\mathcal{C}_n \cup \mathcal{D}_n \cup \bigcup_{i+j=n-1}\mathcal{C}_i\times \mathcal{D}_j\). The \(i\)th boundary map \(d_i : (\mathcal{C} \star \mathcal{D})_n \to (\mathcal{C} \star \mathcal{D})_{n-1}\) is defined on \(\mathcal{C}_n\) and \(\mathcal{D}_n\) using the \(i\)th boundary map on \(\mathcal{C}\) and \(\mathcal{D}\). Given \(\sigma \in S_j, \tau \in T_k\), \(d_i(\sigma, \tau)\) is given by \[\begin{cases}
(d_i \sigma, \tau) & i \leq j,\; j\neq 0,\\
(\sigma, d_{i-j-1}\tau) & i>j, \; k\neq0.
\end{cases}\] If \(j=0\), then \(d_0(\sigma, \tau)=\tau\), and if \(k=0\), then \(d_n(\sigma, \tau)=\sigma\).  (\autocite{htt}, Def 1.2.8.1 / \autocite{join}) & Objects are in both cases disjoint unions of objects from the two categories being joined. Morphisms are also exactly the same in both cases (you get all the morphisms from \(\mathcal{C}\) and \(\mathcal{D}\), plus a morphism from \(c\to d\) for every pair \((c, d)\in \mathcal{C}_0\times \mathcal{D}_0\)). Whenever you have an \(n\)-simplex in \(\mathcal{C}\) and an \(m\)-simplex in \(\mathcal{D}\), you get an \((m+n+1)\)-simplex in \(\mathcal{C}\star \mathcal{D}\), so in particular \(\Delta^n\star \Delta^m \cong \Delta^{m+n+1}\).\\
\hline
Left cone & \(\mathcal{C}^\lhd := [0]\star \mathcal{C}\). & \(\mathcal{C}^\lhd := \Delta^0 \star \mathcal{C}\).  (\autocite{htt}, Not 1.2.8.4) & \(\mathcal{C}\) with extra vertex (cone point) added, as well as a map from that cone point to every other vertex in \(\mathcal{C}\) (plus obligatory degenerate simplicies).\\
\hline
Left Kan extension (along the inclusion of a full subcategory) & Given a commutative diagram \(\begin{tikzcd}
\mathcal{C}^0 \arrow[r, "F_0"] \arrow[d, swap, "\iota", hookrightarrow]  & \mathcal{D} \\
\mathcal{C} \arrow[ur, swap, "F", dashrightarrow]  & 
\end{tikzcd}\), \(F\) is a left Kan extension of \(F_0\) along \(\iota\) if there is a natural transformation \(\eta : F_0 \to F\iota\) such that for any other pair \((G : \mathcal{C} \to \mathcal{D}, \gamma : F_0 \to G\iota)\), there exists a unique natural transformation \(\alpha : F \to G\) such that \(\gamma=(\alpha * \iota)\circ \eta\).
 (\autocite{context}, Def 6.1.1) & Given a commutative diagram \(\begin{tikzcd}
\mathcal{C}^0 \arrow[r, "F_0"] \arrow[d, swap, "\iota", hookrightarrow]  & \mathcal{D} \\
\mathcal{C} \arrow[ur, swap, "F", dashrightarrow]  & 
\end{tikzcd}\), \(F\) is a left Kan extension of \(F_0\) along \(\iota\) if for all \(C \in \mathcal{C}\), the induced diagram \(\begin{tikzcd}
\mathcal{C}^0_{/C} \arrow[r, "F_C"] \arrow[d, swap, "", hookrightarrow]  & \mathcal{D} \\
(\mathcal{C}^0_{/C})^\rhd \arrow[ur, swap, "", dashrightarrow]  &
\end{tikzcd}\) exhibits \(FC\) as a colimit of \(F_C\). (\autocite{htt}, Def 4.3.2.2) & \textcolour{red}{[todo]} \\
\hline
Limit & A limit for \(F : J \to \mathcal{C}\) is a terminal cone on \(F\). & A limit for \(F : X\to \mathcal{C} \) (\(X\) a simplicial set, \(\mathcal{C}\) an \(\infty\)-category) is a final object of \(\mathcal{C}_{/F}\). (\autocite{htt}, Def 1.2.13.4) & The obvious extension of the definition of the overcategory \(\mathcal{C}_{/C}\) for \(C : \{*\} \to \mathcal{C}\) to \(\mathcal{C}_{/F}\)  for an arbitrary functor \(F : J \to \mathcal{C}\) ends up being exactly \(\textbf{Cone}(F)\).\\
\hline
Locally small category & \textcolour{red}{[todo]} & \textcolour{red}{[todo]} & \textcolour{red}{[todo]} \\
\hline
Monoidal category & \textcolour{red}{[todo]} & \textcolour{red}{Cocartesian fibration of \(\infty\)-operads \(\mathcal{C}^\otimes \to \textbf{Assoc}^\otimes\). (\autocite{ha}, Def 4.1.1.10)} & \textcolour{red}{[todo]}\\
\hline
(Coloured) operad & \textcolour{red}{[todo]} & \textcolour{red}{[todo]} & \textcolour{red}{[todo]}\\
\hline
 Opposite category& \(\mathcal{C}^\text{op}\) has the same objects as \(\mathcal{C}\), and \(\Hom_{\mathcal{C}^\text{op}}(X, Y)=\Hom_\mathcal{C}(Y,X)\).  & \(\mathcal{C}^\text{op}_n=\mathcal{C}([n]^\text{op}) \), where \(\{0<1<...<n\}^\text{op}=\{0>1>...>n\}\). (\autocite{htt}, 1.2.1) & A map \(x \to y\) is an edge \(\Delta^1 \to \mathcal{C}\) where \(0\mapsto x\) and \(1 \mapsto y\). In \(\mathcal{C}^\text{op}\) 0 and 1 swap roles, so we instead get a map \(y \to x\).\\
 \hline
  Overcategory & For \(C \in \mathcal{C}, \) the category \(\mathcal{C}_{/C}\) satisfies the following universal property: for any category \(\mathcal{D}\), there is a bijection \[\Hom(\mathcal{D}, \mathcal{C}_{/C})\simeq \Hom_C(\mathcal{D}\star[0], \mathcal{C}),\] where the subscript on the right indicates that we consider only those functors \(\mathcal{D}\star[0] \to \mathcal{C}\) whose restriction to \([0]\) consides with \(C\). (\autocite{htt}, 1.2.9)&  For \(f : S \to \mathcal{C}, \) \(S\) a simplicial set and \(\mathcal{C}\) an \(\infty\)-category, the \(\infty\)-category \(\mathcal{C}_{/f}\) satisfies the following universal property: for any simplicial set \(X\), there is a bijection \[\Hom(X, \mathcal{C}_{/f})\simeq \Hom_f(X\star S, \mathcal{C}),\] where the subscript on the right indicates that we consider only those functors \(X\star S \to \mathcal{C}\) whose restriction to \(S\) consides with \(f\). Explicitly, \[(\mathcal{C}_{/f})_n:=\Hom_f(\Delta^n\star S, \mathcal{C}).\]  (\autocite{htt}, Prop 1.2.9.2) & If \(S = \Delta^0\), writing \(C\in \mathcal{C}\) for the object picked out by \(f\), we have \((\mathcal{C}_{/C})_n=\Hom_C(\Delta^n\star\Delta^0, \mathcal{C})\cong\Hom_C(\Delta^{n+1}, \mathcal{C})\) (where the subscript indicates that we only consider morphisms sending the \((n+1)\)st vertex to \(C\)). In other words, the objects are maps to \(C\), the morphisms are commuting triangles over \(C\), and so on; these are exactly the objects and morphisms in the 1-categorical case. \\
 \hline
 Presentable category & \textcolour{red}{[todo]} & \textcolour{red}{[todo]} & \textcolour{red}{[todo]}\\
 \hline
Presheaf & \textcolour{red}{[todo]} & \textcolour{red}{[todo]} & \textcolour{red}{[todo]}\\
 \hline
Representable functor & \textcolour{red}{[todo]} & \textcolour{red}{[todo]} & \textcolour{red}{[todo]}\\
\hline
Right cone & \(\mathcal{C}^\rhd :=\mathcal{C}\star [0]\). & \(\mathcal{C}^\rhd := \mathcal{C} \star \Delta^0\). (\autocite{htt}, Not 1.2.8.4) & \(\mathcal{C}\) with extra vertex (cone point) added, as well as a map from every other vertex in \(\mathcal{C}\) to that cone point (plus obligatory degenerate simplicies).\\
 \hline
 Subcategory & Subcategory \(\mathcal{C}' \subq \mathcal{C}\). & Subsimplicial set \(\mathcal{C}' \subq \mathcal{C}\) arising as a pullback \(\begin{tikzcd}
\mathcal{C}' \arrow[d, ""'] \arrow[r, ""] \arrow[dr, phantom, "\scalebox{1.3}{$\lrcorner$}" {xshift=-16pt, yshift=6pt}] & \mathcal{C} \arrow[d, ""] \\
N(\text{h}\mathcal{C})' \arrow[r, ""'] & N(\text{h}\mathcal{C})
\end{tikzcd}\) where (h\(\mathcal{C})'\subq\) h\(\mathcal{C}\) is a subcategory. (\autocite{htt}, 1.2.11) &  \textcolour{red}{[todo]}\\
 \hline
 Symmetric monoidal category & \textcolour{red}{[todo]} & \textcolour{red}{[todo]} & \textcolour{red}{[todo]}\\
 \hline
 Symmetric monoidal functor & \textcolour{red}{[todo]} & \textcolour{red}{[todo]}& \textcolour{red}{[todo]}\\
 \hline
 Topos & \textcolour{red}{[todo]} & \textcolour{red}{[todo]} & \textcolour{red}{[todo]} \\
 \hline
 Undercategory & For \(C \in \mathcal{C}, \) the category \(\mathcal{C}_{C/}\) satisfies the following universal property: for any category \(\mathcal{D}\), there is a bijection \[\Hom(\mathcal{D}, \mathcal{C}_{C/})\simeq \Hom_C([0]\star\mathcal{D}, \mathcal{C}),\] where the subscript on the right indicates that we consider only those functors \([0]\star\mathcal{D} \to \mathcal{C}\) whose restriction to \([0]\) consides with \(C\). (\autocite{htt}, 1.2.9) &  For \(f : S \to \mathcal{C}, \) \(S\) a simplicial set and \(\mathcal{C}\) an \(\infty\)-category, the \(\infty\)-category \(\mathcal{C}_{f/}\) satisfies the following universal property: for any simplicial set \(X\), there is a bijection \[\Hom(X, \mathcal{C}_{f/})\simeq \Hom_f(S\star X, \mathcal{C}),\] where the subscript on the right indicates that we consider only those functors \(S\star X \to \mathcal{C}\) whose restriction to \(S\) consides with \(f\). Explicitly, \[(\mathcal{C}_{f/})_n:=\Hom_f(S\star\Delta^n, \mathcal{C}).\]  (\autocite{htt}, Prop 1.2.9.2) & If \(S = \Delta^0\), writing \(C\in \mathcal{C}\) for the object picked out by \(f\), we have \((\mathcal{C}_{C/})_n=\Hom_C(\Delta^0\star\Delta^n, \mathcal{C})\cong\Hom_C(\Delta^{n+1}, \mathcal{C})\) (where the subscript indicates that we only consider morphisms sending the \(0\)th vertex to \(C\)). In other words, the objects are maps from \(C\), the morphisms are commuting triangles under \(C\), and so on; these are exactly the objects and morphisms in the 1-categorical case.\\
\hline
\end{longtable}

\begin{tabular}{ |p{5cm}||p{5cm}|p{7cm}|}
 \hline
 \multicolumn{3}{|c|}{Equivalences}\\
 \hline
 Name& Between & Definition\\
  \hline\hline
    Strong equivalence & Topological categories \(\mathcal{C}, \mathcal{D}\) & \(\mathcal{C} \to \mathcal{D}\) is an equivalnce in the sense of enriched category theory. (\autocite{htt}, Def 1.1.3.1)\\
 \hline
  (Weak) equivalence & Topological categories \(\mathcal{C}, \mathcal{D}\) & The induced functor h\(\mathcal{C} \to\) h\(\mathcal{D}\) is an equivalence of \(\mathcal{H}\)-enriched categories. (\autocite{htt}, Def 1.1.3.6)\\
  \hline
  Categorical equivalence & Simplicial sets \(X, S\) & The induced functor h\(X \to\) h\(S\) is an equivalence of \(\mathcal{H}\)-enriched categories. (\autocite{htt}, Def 1.1.5.14)\\
\hline 
 Weak (homotopy) equivalence & Simplicial sets \(X, S\) & The induced map \(\abs{X} \to \abs{S}\) is a weak homotopy equivalence of topological spaces. (\autocite{htt}, 1.1.4)\\
 \hline
 Equivalence & Simplicial categories \(\mathcal{C}, \mathcal{D}\) &The induced functor h\(\mathcal{C} \to\) h\(\mathcal{D}\) is an equivalence of \(\mathcal{H}\)-enriched categories. (\autocite{htt}, Def 1.1.4.4)\\
\hline
\end{tabular}

\text{}

\text{}

\begin{longtable}{ |p{5cm}||p{5cm}|p{7cm}| }
 \hline
 \multicolumn{3}{|c|}{Fibrations and anodyne morphisms}\\
 \hline
 Name& Describes & Definition\\
  \hline\hline
  Acyclic Kan fibration & \(f : X \to S\) map of simplicial sets & see: trivial Kan fibration. (\autocite{acyclic})\\
  \hline
Anodyne & \(f : X \to S \) map of simplicial sets & For every solid arrow diagram as below, with \(p : Y \to T\) a Kan fibration, \[\begin{tikzcd}
X \arrow[r, ""] \arrow[d, swap, "f"]  & Y \arrow[d, "p"]  \\
S \arrow[r, swap, ""] \arrow[ur, dashrightarrow]  & T
\end{tikzcd}\] there exists a dotted lift. (\autocite{htt}, Ex 2.0.0.1) \\
\hline 
   Cartesian fibration & \(f : X \to S\) map of simplicial sets & \(f\) is an inner fibration such that for every edge \(g : x \to y\) of \(S\) and every vertex \(\tilde y\) of \(X\) with \(f(\tilde y)=y\), there exists an \(f\)-cartesian edge \(\tilde g : \tilde x \to \tilde y\) with \(f(\tilde g)=g\). (\autocite{htt}, Def 2.4.2.1)\\
 \hline
Categorical fibration & \(f : X \to S\) map of simplicial sets & For every solid arrow diagram as below, with \(p : Y \to T\) both a  cofibration and a categorical equivalence, \[\begin{tikzcd}
Y \arrow[r, ""] \arrow[d, swap, "p"]  & X \arrow[d, "f"]  \\
T \arrow[r, swap, ""] \arrow[ur, dashrightarrow]  & S
\end{tikzcd}\] there exists a dotted lift. (\autocite{htt}, p90)\\
\hline
    Cocartesian fibration & \(f : X \to S\) map of simplicial sets &\(f\) is an inner fibration such that for every edge \(g : x \to y\) of \(S\) and every vertex \(\tilde x\) of \(X\) with \(f(\tilde x)=x\), there exists an \(f\)-cocartesian edge \(\tilde g : \tilde x \to \tilde y\) with \(f(\tilde g)=g\). (\autocite{htt}, Def 2.4.2.1)\\
 \hline
 Cofibration & \(f : X\to S \) map of simplicial sets & \(f\) is a monomorphism. (\autocite{htt}, A.2.7) \\
 \hline
  Inner anodyne & \(f : X \to S \) map of simplicial sets & For every solid arrow diagram as below, with \(p : Y \to T\) an inner fibration, \[\begin{tikzcd}
X \arrow[r, ""] \arrow[d, swap, "f"]  & Y \arrow[d, "p"]  \\
S \arrow[r, swap, ""] \arrow[ur, dashrightarrow]  & T
\end{tikzcd}\] there exists a dotted lift. (\autocite{htt}, Def 2.0.0.3) \\
\hline 
  Inner fibration & \(f : X\to S \) map of simplicial sets & For every solid arrow diagram as below, with \(0 < i < n\), \[\begin{tikzcd}
\Lambda^n_i \arrow[r, ""] \arrow[d, swap, hookrightarrow]  & X \arrow[d, "f"]  \\
\Delta^n \arrow[r, swap, ""] \arrow[ur, dashrightarrow]  & S
\end{tikzcd}\] there exists a dotted lift.\\
 \hline
 Isofibration & \(F : \mathcal{C} \to \mathcal{D}\) map of \(\infty\)-categories &\(F\) is an inner fibration such that for all \(C \in \mathcal{C}\) and every isomorphism \(u : D \to FC\) in \(\mathcal{D}\) (i.e. \([u]\) is an isomorphism in h\(\mathcal{D}\)) there exists an isomorphism \(\overline u : \overline D \to C\) in \(\mathcal{C}\) such that \(F(\overline u)=u\). \cite[\href{https://kerodon.net/tag/01EN}{Def 01EN}]{kerodon}\\
 \hline
(Kan) fibration & \(f : X\to S \) map of simplicial sets & For every solid arrow diagram as below, with \(0 \leq i \leq n\), \[\begin{tikzcd}
\Lambda^n_i \arrow[r, ""] \arrow[d, swap, hookrightarrow]  & X \arrow[d, "f"]  \\
\Delta^n \arrow[r, swap, ""] \arrow[ur, dashrightarrow]  & S
\end{tikzcd}\] there exists a dotted lift. (\autocite{htt}, A.2.7) \\
 \hline
   Left anodyne & \(f : X \to S \) map of simplicial sets & For every solid arrow diagram as below, with \(p : Y \to T\) a left fibration, \[\begin{tikzcd}
X \arrow[r, ""] \arrow[d, swap, "f"]  & Y \arrow[d, "p"]  \\
S \arrow[r, swap, ""] \arrow[ur, dashrightarrow]  & T
\end{tikzcd}\] there exists a dotted lift. (\autocite{htt}, Def 2.0.0.3) \\
\hline 
 Left fibration & \(f : X \to S \) map of simplicial sets & For every solid arrow diagram as below, with \(0 \leq i < n\), \[\begin{tikzcd}
\Lambda^n_i \arrow[r, ""] \arrow[d, swap, hookrightarrow]  & X \arrow[d, "f"]  \\
\Delta^n \arrow[r, swap, ""] \arrow[ur, dashrightarrow]  & S
\end{tikzcd}\] there exists a dotted lift. (\autocite{htt}, Def 2.0.0.3) \\
\hline 
 Right anodyne & \(f : X \to S \) map of simplicial sets & For every solid arrow diagram as below, with \(p : Y \to T\) a right fibration, \[\begin{tikzcd}
X \arrow[r, ""] \arrow[d, swap, "f"]  & Y \arrow[d, "p"]  \\
S \arrow[r, swap, ""] \arrow[ur, dashrightarrow]  & T
\end{tikzcd}\] there exists a dotted lift. (\autocite{htt}, Def 2.0.0.3) \\
\hline 
  Right fibration & \(f : X \to S \) map of simplicial sets & For every solid arrow diagram as below, with \(0 < i \leq n\), \[\begin{tikzcd}
\Lambda^n_i \arrow[r, ""] \arrow[d, swap, hookrightarrow]  & X \arrow[d, "f"]  \\
\Delta^n \arrow[r, swap, ""] \arrow[ur, dashrightarrow]  & S
\end{tikzcd}\] there exists a dotted lift. (\autocite{htt}, Def 2.0.0.3) \\
\hline 
 Serre fibration & \(f : Y \to Z\) map of topological spaces & For every solid arrow diagram as below, \[\begin{tikzcd}
\{0\}\times \abs{\Delta^n} \arrow[r, ""] \arrow[d, swap]  & Y \arrow[d, "f"]  \\
\text{[}0,1\text{]}\times \abs{\Delta^n} \arrow[r, swap, ""] \arrow[ur, dashrightarrow]  & Z
\end{tikzcd}\] there exists a dotted lift. \cite[\href{https://kerodon.net/tag/021R}{Def 021R}]{kerodon}\\
\hline 
 Trivial (Kan) fibration & \(f : X \to S\) map of simplicial sets & For every solid arrow diagram as below, \[\begin{tikzcd}
\del\Delta^n \arrow[r, ""] \arrow[d, swap, hookrightarrow]  & X \arrow[d, "f"]  \\
\Delta^n \arrow[r, swap, ""] \arrow[ur, dashrightarrow]  & S
\end{tikzcd}\] there exists a dotted lift. (\cite[\href{https://kerodon.net/tag/006W}{Def 006W}]{kerodon}/\autocite{htt}, Def 2.0.0.2)\\
\hline 
\end{longtable}

\text{}

\text{}

\begin{tabular}{ |p{3cm}||p{5cm}|p{7cm}|}
 \hline
 \multicolumn{3}{|c|}{Nerves}\\
 \hline
 Name& Domain object & Definition\\
  \hline\hline
  Nerve & Category \(\mathcal{C}\) & \((N\mathcal{C})_n=\text{\{}n\text{-composable strings of morphisms}\) \(\text{ in } \mathcal{C}\text{\}}\). \\
 \hline
 Simplicial nerve & Simplicial category \(\mathcal{C}\) & \((N \mathcal{C})_n=\Hom_{\textbf{Cat}_\Delta}(\mathfrak{C}[\Delta^n], \mathcal{C})\), where \(\mathfrak{C}[\Delta^n]\) is the category whose objects are the same as \([n]\), and \(\Hom_{\mathfrak{C}[\Delta^n]}(i,j)=\emptyset\) for \(i<j\) and \(N(P_{ij})\) for \(i\geq j\) (where \(P_{ij}=\{I \subq [n] : (i, j \in I)\wedge (\forall k \in I, i \leq k \leq j)\}\)).\\
 \hline
 Topological nerve & Topological category \(\mathcal{C}\) & The simplicial nerve of \(\Sing \mathcal{C}\).\\
 \hline
\end{tabular}

\text{}

\text{}

\begin{tabular}{ |p{5cm}||p{8cm}|}
 \hline
 \multicolumn{2}{|c|}{Homotopy categories}\\
 \hline
 Domain object & Definition\\
  \hline\hline
  \(\infty\)-Category \(\mathcal{C}\) & The objects of h\(\mathcal{C}\) are the vertices of \(\mathcal{C}\), and \(\Hom_{\text{h}\mathcal{C}}(X,Y)\) is the set of homotopy classes of edges \(X\to Y\) in \(\mathcal{C}\). (\autocite{htt}, Prop 1.2.3.9)\\
  \hline
  Simplicial category \(\mathcal{C}\) & h\(\abs{\mathcal{C}}\). (\autocite{htt}, 1.1.4) \\ 
 \hline
   Topological category \(\mathcal{C}\) & h\(\mathcal{C}\) has the same objects as \(\mathcal{C}\), and \(\Hom_{\text{h}\mathcal{C}}(X,Y)=[\Hom_\mathcal{C}(X,Y)]\). (\autocite{htt}, 1.1.3) \\
  \hline 
\end{tabular}

\text{}

\text{}

\begin{tabular}{ |p{5cm}||p{8cm}|}
 \hline
 \multicolumn{2}{|c|}{Categories}\\
 \hline
 Name & Definition\\
  \hline\hline
\(\textbf{Assoc}^\otimes\) & \textcolour{red}{[todo]} \\
\hline
\textbf{Kan} & The full subcategory of \textbf{sSet} spanned by the collection of small Kan complexes. (\autocite{htt}, Def 1.2.16.1) \\ 
 \hline
\textbf{KAN} & The category of all Kan complexes. (\autocite{htt}, Rem 5.1.6.1)\\
\hline
  \(\mathcal{S}\) (the \(\infty\)-category of spaces) & The simplicial\footnotemark nerve \(N(\textbf{Kan})\).\\
  \hline 
\(\widehat{\mathcal{S}}\) & The simplicial nerve \(N(\textbf{KAN})\).\\
\hline
\end{tabular}
\footnotetext{{\textbf{sSet} is a simplicial category, with \(\Hom(X, S)_n=\Hom_{\textbf{sSet}}(\Delta^n \times X, S)\). The subcategory \textbf{Kan} inherits this structure.}}
% what is LaTeX's problem?? How hard is it to let someone put a footnote in a table?

\end{centre}

\newgeometry{top=2cm, bottom=2cm, left=2cm, right=2cm}
\printbibliography
\restoregeometry%
\clearpage

\end{document}
